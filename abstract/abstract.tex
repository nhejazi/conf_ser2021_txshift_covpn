\documentclass[]{elsarticle} %review=doublespace preprint=single 5p=2 column
%%% Begin My package additions %%%%%%%%%%%%%%%%%%%
\usepackage[hyphens]{url}

  \journal{the 2020 SER Meeting} % Sets Journal name


\usepackage{lineno} % add
\providecommand{\tightlist}{%
  \setlength{\itemsep}{0pt}\setlength{\parskip}{0pt}}

\usepackage{graphicx}
\usepackage{booktabs} % book-quality tables
%%%%%%%%%%%%%%%% end my additions to header

\usepackage[T1]{fontenc}
\usepackage{lmodern}
\usepackage{amssymb,amsmath}
\usepackage{ifxetex,ifluatex}
\usepackage{fixltx2e} % provides \textsubscript
% use upquote if available, for straight quotes in verbatim environments
\IfFileExists{upquote.sty}{\usepackage{upquote}}{}
\ifnum 0\ifxetex 1\fi\ifluatex 1\fi=0 % if pdftex
  \usepackage[utf8]{inputenc}
\else % if luatex or xelatex
  \usepackage{fontspec}
  \ifxetex
    \usepackage{xltxtra,xunicode}
  \fi
  \defaultfontfeatures{Mapping=tex-text,Scale=MatchLowercase}
  \newcommand{\euro}{€}
\fi
% use microtype if available
\IfFileExists{microtype.sty}{\usepackage{microtype}}{}
\bibliographystyle{elsarticle-harv}
\usepackage{graphicx}
% We will generate all images so they have a width \maxwidth. This means
% that they will get their normal width if they fit onto the page, but
% are scaled down if they would overflow the margins.
\makeatletter
\def\maxwidth{\ifdim\Gin@nat@width>\linewidth\linewidth
\else\Gin@nat@width\fi}
\makeatother
\let\Oldincludegraphics\includegraphics
\renewcommand{\includegraphics}[1]{\Oldincludegraphics[width=\maxwidth]{#1}}
\ifxetex
  \usepackage[setpagesize=false, % page size defined by xetex
              unicode=false, % unicode breaks when used with xetex
              xetex]{hyperref}
\else
  \usepackage[unicode=true]{hyperref}
\fi
\hypersetup{breaklinks=true,
            bookmarks=true,
            pdfauthor={},
            pdftitle={2020 SER Meeting Abstract},
            colorlinks=false,
            urlcolor=blue,
            linkcolor=magenta,
            pdfborder={0 0 0}}
\urlstyle{same}  % don't use monospace font for urls

\setcounter{secnumdepth}{0}
% Pandoc toggle for numbering sections (defaults to be off)
\setcounter{secnumdepth}{0}


% Pandoc header



\begin{document}
\begin{frontmatter}

  \title{2020 SER Meeting Abstract}
    \author[biostat-berkeley]{Nima S. Hejazi}
  
    \author[epibiostat-berkeley]{Mark J. van der Laan}
  
    \author[fredhutch]{Holly E. Janes}
  
    \author[fredhutch]{Peter B. Gilbert}
  
    \author[emory]{David C. Benkeser}
  
      \address[biostat-berkeley]{Graduate Group in Biostatistics, University of California, Berkeley}
    \address[epibiostat-berkeley]{Division of Epidemiology and Biostatistics, University of California,
Berkeley}
    \address[fredhutch]{Vaccine and Infectious Disease Division, Fred Hutchinson Cancer Research
Center}
    \address[emory]{Department of Biostatistics and Bioinformatics, Emory University}
      \cortext[1]{Character count: 1976 (with spaces)}
    \cortext[2]{Corresponding author: nhejazi@berkeley.edu}
    \cortext[3]{Key words: causal inference, targeted learning, two-phase sampling,
stochastic intervention, vaccine efficacy}
    \cortext[4]{Title: Efficient estimation of stochastic interventions effects under
two-phase sampling for the analysis of vaccine efficacy trials}
  
  \begin{abstract}
  
  \end{abstract}
  
 \end{frontmatter}

Causal inference has classically focused on the effect of static
interventions, under which, for each unit, the magnitude of the
treatment is set to a fixed, prespecified value. The evaluation of such
interventions faces a host of issues, among them non-identification,
violations of the assumption of positivity, and inefficiency. Stochastic
interventions provide a promising solution to these fundamental issues
of causal inference, by allowing for the counterfactual intervention
distribution to be defined as a function of its natural (observed)
distribution. Despite the promise of such approaches, real data analyses
are often further complicated by economic constraints, such as when the
primary variable of interest is far more expensive to collect than
auxiliary covariates. Two-phase sampling schemes are often used to work
around such constraints -- unfortunately, their use produces side
effects that require further adjustment when formal inference is the
principal goal of a study. We present a novel approach for use in such
settings: augmented targeted minimum loss and one-step estimators for
the causal effects of stochastic interventions, with guarantees of
consistency, efficiency, and multiple robustness even in the presence of
two-phase sampling. We further propose a technique that utilizes
estimated counterfactual means under stochastic interventions to
construct a nonparametric working marginal structural model to summarize
the effects of changes in an exposure variable on the outcome of
interest, analogous to a dose-response analysis. Using data from the
recent HVTN 505 HIV vaccine efficacy trial, we demonstrate this
technique by assessing the effects of changes in post-vaccination
immunogenicity on HIV-1 acquisition across a range of possible shifts,
outlining a highly interpretable variable importance measure for ranking
multiple immune responses based on their utility as immunogenicity study
endpoints in future HIV-1 vaccine trials.

\newpage

\includegraphics{./cd8_msm_tmle_summary.pdf}


\end{document}


